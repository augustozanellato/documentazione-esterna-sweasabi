\subsection{Capitolato C3: \textit{Personal Identity Wallet}}
\subsubsection{Informazioni generali}
\textbf{Proponente:} \textit{Infocert S.p.A.}

\textbf{Descrizione:} Ci viene richiesto di sviluppare un progetto che aiuti a verificare le credenziali di una persona, attraverso un wallet personale.

Chi crea e firma digitalmente le credenziali è detto \textit{issuer}. Queste credenziali vengono poi date all' \textit{holder}, ovvero l'utente generico, che le gestirà attraverso una wallet personale.

Successivamente, l'\textit{holder} potrà presentare le credenziali ad un \textit{verifier} per attestare ad esempio un titolo di studio.

Il \textit{verifier} dovrà poi verificare l'autenticità delle credenziali consultando un \textit{verifiable data registry}, ovvero un'infrastruttura gestita dall'\textit{issuer}.

\subsubsection{Dominio}
\textbf{Dominio applicativo}

Il capitolato vuole sviluppare un'applicazione web responsive dotata delle seguenti funzioni:
\begin{itemize}
    \item richiesta di credenziali da parte dell'holder;
    \item autorizzazione per la creazione delle credenziali da parte dell'issuer;
    \item possibilità dell'holder di fornire le proprie credenziali al verifier.
\end{itemize}
Inoltre, il capitolato vuole sviluppare una seconda applicazione web responsive per permettere all'holder di gestire le proprie credenziali.

\textbf{Tecnologie}

Per lo sviluppo delle applicazioni non è ancora chiaro che tecnologie si utilizzeranno, perché verranno fornite da \textit{Infocert S.p.A.} dopo la scelta del capitolato.

\subsubsection{Fattori determinanti}
\begin{itemize}
    \item Progetto in linea con una ottica futura del mercato;
    \item complessità elevata del progetto.
\end{itemize}
\subsubsection{Decisione}
Il capitolato ci è parso molto interessante, ma la sua difficoltà e il fatto di non sapere esattamente quali tecnologie si dovranno utilizzare ci ha portato a \textbf{rifiutare il progetto}.
