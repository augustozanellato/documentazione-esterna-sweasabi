\section{Punti del giorno e aspettative}

\begin{enumerate}
    \item \textbf{Riguardo agli incontri periodici e ai metodi di comunicazione preferiti}
    \begin{itemize}
        \item L'azienda propone incontri bisettimanali/settimanali ufficiali dove verrà presentato il lavoro svolto fino a quel momento (di conseguenza si svilupperà un verbale sulla riunione);
        \item in ogni caso il proponente si mette a disposizione per incontri aggiuntivi in chiamata, usando un server sull'applicativo Discord di proprietà dell'azienda o per messaggio, su Telegram, per questioni risolvibili velocemente;
    \end{itemize}
    \item \textbf{Figure affiancate durante lo svolgimento del progetto}
        \begin{itemize}
            \item Il proponente ha messo a nostra disposizione in caso di necessità un team composto da:
            \begin{itemize}
                \item una persona con 10 anni di esperienza su blockchain: Matteo Galvagni;
                \item una persona con 20 anni di esperienza: Fabio Pallaro;
                \item una persona con 2/3 anni di esperienza su Angular.
            \end{itemize}
        \end{itemize}
    \item \textbf{Servizi consigliati}
        \begin{itemize}
            \item I servizi da utilizzare sono stati suddivisi secondo il proponente in "consigli" e "forti consigli":
            \begin{itemize}
                \item consigli:
                \begin{itemize}
                    \item Angular è un consiglio per lo sviluppo della WebApp: è considerato il migliore vista la sua diffusione, in particolare nell'ambito dello sviluppo di soluzioni più complicate.
                
                    L'azienda usa Angular ma è possibile utilizzare anche React o altri framework (anche se questi ultimi sono più adatti per applicazioni meno complesse);
                    \item evitare assolutamente framework nuovi, solitamente non dispongono di una documentazione sufficientemente dettagliata.
                \end{itemize}
                \item{forti consigli:}
                \begin{itemize}
                    \item Blockchain ethereum compatibile: si può scegliere una qualsiasi blockchain che sia retrocompatibile con Ethereum.
                    \item Solidity per la definizione degli Smart Contracts.
                    \item Metamask come wallet per intefacciare con le webapp per due motivi. Il primo, uno dei più grandi ed usati e di conseguenza è disponibile molta più documentazione a riguardo (semistandard), il secondo, Matteo Galvagni ha esperienza a riguardo.
                \end{itemize}
            \end{itemize}
        \end{itemize}
    \item \textbf{Servizi di testing e coverage}
        \begin{itemize}
            \item test end-to-end e test di unità (solo coverage e nessun grafico) fanno da requisito minimo per passare il progetto;
            \item test con Truffle per la parte in JavaScript;
            \item per i test della parte in Angular saranno forniti dei file appositi.
            \item servizio di coverage a nostra discrezione, si può discuterne a riguardo, in ogni caso è essenziale che venga mostrata la percentuale.
        \end{itemize}
    \item \textbf{Le transazioni avverranno solo su blockchain?}
        \begin{itemize}
            \item Le transazioni economiche dovranno avvennire solo su contratto digitale (quindi blockchain), in particolare si è accennato all'uso del valore stabile.
        \end{itemize}
\end{enumerate}





